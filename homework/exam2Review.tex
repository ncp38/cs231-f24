\documentclass[10pt]{article}
\usepackage[top=1in,bottom=1.1in,left=.8in,right=.8in]{geometry}
\usepackage[T1]{fontenc}
\usepackage[ansinew]{inputenc}
\usepackage{graphicx}
\usepackage{multirow}
\usepackage{enumitem}


\renewcommand{\familydefault}{\sfdefault}

\begin{document}
%\maketitle

\hspace{-5mm}
\begin{minipage}{0.65\linewidth}
  \textbf{
      \hspace{-3mm}
      {\Large COMP 231-01}\\
      {\Large Introduction to Computer Organization}\\
      {\Large Exam II Review}}
\end{minipage}
\begin{minipage}{0.35\linewidth}
  \includegraphics[scale=.3]{../../logos/rhodes-logo.jpg}
\end{minipage}

%\noindent{\bf Due Date: Monday, February 13th.}\\
%\noindent{\bf Due Date: Monday, September 19th.}\\
%\noindent{\bf Due Date: Tuesday, September 26th.}\\
%\noindent{\bf Due Date: Thursday, September 27th.}\\
%\noindent{\bf Due Date: Friday, February 8th.}\\
%\noindent{\bf Due Date: Monday, September 16th.}\\



\begin{itemize}

\setlength\itemsep{10mm}


\item Answer the following questions.  For the assembly language questions, assume that all variables are signed integers.  All fragments are to be written as a NIOS II assembly language fragment.



  \begin{enumerate}
\setlength\itemsep{5mm}

\item Write a NIOS II assembly language fragment that reads a 32 bit value from memory location 0x0040, doubles it, and stores the result in r13.

    \vspace{1in}

\item Translate the following into assembly, assuming x is stored in register r11 and y is stored at memory location 0x0084:
    \begin{verbatim}
y = 5;
y++;
x = y + 12;
    \end{verbatim}
    \vspace{1in}

\item Translate the following into assembly, assuming the register-like variables are stored in a register.

r8 = r12 + 20;

    \vspace{1in}

\item Translate the following into assembly, assuming x is stored in register r11 and y is stored at memory location 0x0084:
    \begin{verbatim}
if(x != y)
  x=50;
else
  y=25;
    \end{verbatim}
    \vspace{1in}

\item Translate the following into assembly, assuming x is stored in register r11, y is stored at memory location 0x0084, and z is stored at 0x0200:
    \begin{verbatim}
while(y <= 10 && x > 10)
  y++;
  x--;
    \end{verbatim}
    \vspace{1in}


\item Translate the following into assembly :
    \begin{verbatim}
r8 = 10;
r9 = 0;

while(r8 > 10)
  r9 = r9 + r8;
    \end{verbatim}
    \vspace{1in}

\item Translate the following into assembly:
    \begin{verbatim}
for(int i = 0; i < 6; i++)
  r11 = r11 + i;
    \end{verbatim}
    \vspace{1in}

\item Translate the following into assembly:
    \begin{verbatim}
if(r8 == 0)
  r8 = r12 + 9;
else if(r8 < 100 || r12 > 100)
  r12 = r8;
else
  r8 = 10;
  r12 = 10;
    \end{verbatim}
    \vspace{1in}

\item Assume we have an array, a[], of 20 integers starting at memory address x1000.  Write a program that stores each power of 2 (starting at 1) in the first 8 indices of this array.  Your program should calculate the power of 2 and then store that value in the index. You are free to use your choice of registers, memory space, and labels as needed for your program.\\ 

a[0] will become 2\\
a[1] will become 4\\
a[2] will become 8\\
etc.

    \vspace{1in}

\item Assume we have an array, a[], of 20 integers starting at memory address x1000.  Write a program that compares each index in that array to a variable initialized to 100.  If the value in the index is greater than or equal to the variable, add 10 to the variable.  If the value at the index is less than the variable, subtract 10 from it. You are free to use your choice of registers, memory space, and labels as needed for your program.\\ 

If a[0] is 12, the value changes to 90\\
If a[1] is 120, the value changes to 100\\
If a[2] is 200, the value changes to 110\\
etc.

    \vspace{1in}

\newpage

\item How many bits does a single memory address store?

\item What is the size of a word in NIOS II assembly language?

\item What does the 0x symbolize in the memory address 0x5433?

\item What is the difference between the stw and stwio commands?

\item What is sign extension?  Why would you use sign extension?

\item What is a pseudoinstruction?  In what situation would a pseudoinstruction be used?

\item How many regular registers does NIOS II have?

\item If you were to include a value directly in an assembly command (that is, not from a register or memory location), what addressing mode would you be using?

\item Describe what the following absolute address describes: $20(r12)$

\item Describe what the following absolute address describes: $0(r10)$

\item Describe what the following absolute address describes: $11(r9)$

\item What is the content of Register 0?

\item What happens if you change the value of the program counter?

\item Why can the program counter not be set to an odd number without an error?

\item If you were to look inside the contents of a command in binary, what are three components that you would see?



\end{enumerate}

\end{itemize}

\newpage

\section*{NIOS II Instruction Reference}

This is a list of the syntax for the instructions that you may use
from the NIOS instruction set:\\

\texttt{%
\begin{tabular}{l|l}
  {\bf instruction} \hspace{2in} & \hspace{1in}{\bf usage}\\\hline
  ldw & ldw rB, off(rA)\\
  ldb & ldb rB, off(rA)\\
  ldbu & ldbu rB, off(rA)\\
  ldbu & ldbu rB, off(rA)\\
  ldh & ldh rB, off(rA)\\
  ldhu & ldhu rB, off(rA)\\
  stw & stw rB, off(rA)\\
  stb & stw rB, off(rA)\\
  sth & stw rB, off(rA)\\
  add & add rC, rA, rB\\
  sub & sub rC, rA, rB\\
  mul & mul rC, rA, rB\\
  div & div rC, rA, rB\\
  addi & addi rC, rA, IMMED16\\
  subi & subi rC, rA, IMMED16\\
  muli & muli rC, rA, IMMED16\\
  divu & divu rC, rA, rB\\
  and & and rC, rA, rB\\
  or & or rC, rA, rB\\
  xor & xor rC, rA, rB\\
  andi & andi rC, rA, IMMED16\\
  ori & ori rC, rA, IMMED16\\
  xori & xori rC, rA, IMMED16\\
  andhi & andhi rC, rA, IMMED16\\
  orhi & orhi rC, rA, IMMED16\\
  xorhi & xorhi rC, rA, IMMED16\\
  mov & mov rC, rA\\
  movi & movi rB, IMMED16\\
  movui & movui rB, IMMED16\\
  srl & srl rC, rA, rB\\
  srli & srli rC, rA, IMMED5\\
  sra & sra rC, rA, rB\\
  srai & srai rC, rA, IMMED5\\
  sll & sll rC, rA, rB\\
  slli & slli rC, rA, IMMED5\\
  jmp & jmp rA\\
  br & br LABEL\\
  beq & beq rA, rB, LABEL\\
  bne & bne rA, rB, LABEL\\
  blt & blt rA, rB, LABEL\\
  ble & ble rA, rB, LABEL\\
  bgt & bgt rA, rB, LABEL\\
  bge & bge rA, rB, LABEL\\
  bltu & bltu rA, rB, LABEL\\
  bleu & bleu rA, rB, LABEL\\
  bgtu & bgtu rA, rB, LABEL\\
  bgeu & bgeu rA, rB, LABEL\\
  call & call LABEL\\
  ret & ret\\
\end{tabular}
}

\end{document}


\documentclass[10pt]{article}
\usepackage[top=1in,bottom=1.1in,left=.8in,right=.8in]{geometry}
\usepackage[T1]{fontenc}
\usepackage[ansinew]{inputenc}
\usepackage{graphicx}
\usepackage{multirow}


\renewcommand{\familydefault}{\sfdefault}

\begin{document}
%\maketitle

\hspace{-5mm}
\begin{minipage}{0.65\linewidth}
  \textbf{
      \hspace{-3mm}
      {\Large COMP 231-01}\\
      {\Large Introduction to Computer Organization}\\
      {\Large Homework 5}}
\end{minipage}
\begin{minipage}{0.35\linewidth}
  \includegraphics[scale=.3]{../../logos/rhodes-logo.jpg}
\end{minipage}

\noindent{\bf Due Date: Tuesday, November 19}\\
%\noindent{\bf Due Date: Monday, November 21st.}\\
%\noindent{\bf Due Date: Friday, April 12th.}\\


\begin{enumerate}

\setlength\itemsep{10mm}

\item Consider the following C code:

\begin{verbatim}
  int a[1000];
  main() {

     int r10 = 5;  // lower bound of array slice
     int r16 = 10; // upper bound of array slice
     ...
     int r22 = f(r10, r16);
     ...
     r23 = r10 + r16 + r22;
   }   

   int f(int x, int y) {
      int r10 = 0; // local sum
      int r16 = 1; // local product
      
      for (int r17 = x; r17 < y; r17++) {
         r10 += a[r17];
         r16 *= a[r17];
      }
      return r10 + r16;
}
 
\end{verbatim}

Convert this code to NIOS assembly. Assume that each of the integers
that have register-like names live in the same register. Also assume
that the array {\tt a[]} begins at memory address 1000. Show the code
to allocate a stack frame and preserve both caller-save and
callee-save registers.

\end{enumerate}


\end{document}


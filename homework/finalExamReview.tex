\documentclass[10pt]{article}
\usepackage[top=1in,bottom=1.1in,left=.8in,right=.8in]{geometry}
\usepackage[T1]{fontenc}
\usepackage[ansinew]{inputenc}
\usepackage{graphicx}
\usepackage{multirow}
\usepackage{enumitem}


\renewcommand{\familydefault}{\sfdefault}

\begin{document}
%\maketitle

\hspace{-5mm}
\begin{minipage}{0.65\linewidth}
  \textbf{
      \hspace{-3mm}
      {\Large COMP 231-01}\\
      {\Large Introduction to Computer Organization}\\
      {\Large Final Exam Review}}
\end{minipage}
\begin{minipage}{0.35\linewidth}
  \includegraphics[scale=.3]{../../logos/rhodes-logo.jpg}
\end{minipage}

%\noindent{\bf Due Date: Monday, February 13th.}\\
%\noindent{\bf Due Date: Monday, September 19th.}\\
%\noindent{\bf Due Date: Tuesday, September 26th.}\\
%\noindent{\bf Due Date: Thursday, September 27th.}\\
%\noindent{\bf Due Date: Friday, February 8th.}\\
%\noindent{\bf Due Date: Monday, September 16th.}\\



For this exam, the main addition to the exam is \textbf{functions} in assembly language and the register conventions we talked about in class.  When you're designing assembly language programs on the test, you're responsible for using registers correctly according to the conventions.  I recommend studying and reviewing the \textbf{register conventions} section of the assembly textbook and then working on coding the following exercises.  We're also covering the FDXW cycle and memory, but those questions will appear as short answer problems. 

To study for this exam, I recommend reviewing your work for the course, most especially the exam reviews and tests, as well as this final exam review.  This exam is cumulative and will include questions from across the semester. 

\begin{itemize}

\setlength\itemsep{10mm}


\item The function detailed below is a recursive C function for calculating the result of an exponentiation operation.  Note that this function is designed for integers greater than or equal to 0.
    \begin{verbatim}
int exponentiate(int base, int power)
{
         if(power <= 0)
         {
                  return 1;
         }
         else
         {
                  return base * exponentiate(base, power-1);
         }
}
    \end{verbatim}

Translate this function into assembly language, using the calling conventions.  
    \vspace{3in}
\newpage
\item The code below is an assembly language program that has been improperly translated and does not follow the calling conventions.  Find and correct the errors in this program.  (There are intended to be 8 errors.  Some will be related to the calling conventions, but others will be related to other aspects of the program.)
    \begin{verbatim}
#This function takes the values in memory location 1000 and applies the MATHFUNC to it.
#Then, this function adds that same value (from MEM 1000) with the results of the MATHFUNC operation.
_start
ldw r9, 1000

call MATHFUNC

add r0, r9, r2

MATHFUNC: #This function triples the argument value and adds 2 to it, then returns that value.
call FUNC
addi r9, 2
ret


FUNC:  #This function takes in an argument, triples the given value, and returns that value.
multi r9, r9, 3
    \end{verbatim}

Translate this function into assembly language, using the calling conventions. 


    \vspace{1in}

\item The function detailed below is a recursive C function for calculating the result of a factorial operation.  Note that this function is designed for integers greater than or equal to 0.
    \begin{verbatim}
int factorial(int input)
{
         if(input <= 1)
         {
                  return 1;
         }
         else
         {
                  return base * factorial(input - 1);
         }
}
    \end{verbatim}

Write an assembly program to calculate the factorial value of 10.  To do this, write the main function and also translate this into assembly language, using the calling conventions.  
    \vspace{3in}

\newpage


\item What is meant by the term spatial locality?

\item What is meant by the term temporal locality?

\item What is meant by the term pipelining?

\item Why is out of order processing used in assembly language?

\item What is the inherent conflict between branches and pipelining?

\end{itemize}

\newpage

\section*{NIOS II Instruction Reference}

This is a list of the syntax for the instructions that you may use
from the NIOS instruction set:\\

\texttt{%
\begin{tabular}{l|l}
  {\bf instruction} \hspace{2in} & \hspace{1in}{\bf usage}\\\hline
  ldw & ldw rB, off(rA)\\
  ldb & ldb rB, off(rA)\\
  ldbu & ldbu rB, off(rA)\\
  ldbu & ldbu rB, off(rA)\\
  ldh & ldh rB, off(rA)\\
  ldhu & ldhu rB, off(rA)\\
  stw & stw rB, off(rA)\\
  stb & stw rB, off(rA)\\
  sth & stw rB, off(rA)\\
  add & add rC, rA, rB\\
  sub & sub rC, rA, rB\\
  mul & mul rC, rA, rB\\
  div & div rC, rA, rB\\
  addi & addi rC, rA, IMMED16\\
  subi & subi rC, rA, IMMED16\\
  muli & muli rC, rA, IMMED16\\
  divu & divu rC, rA, rB\\
  and & and rC, rA, rB\\
  or & or rC, rA, rB\\
  xor & xor rC, rA, rB\\
  andi & andi rC, rA, IMMED16\\
  ori & ori rC, rA, IMMED16\\
  xori & xori rC, rA, IMMED16\\
  andhi & andhi rC, rA, IMMED16\\
  orhi & orhi rC, rA, IMMED16\\
  xorhi & xorhi rC, rA, IMMED16\\
  mov & mov rC, rA\\
  movi & movi rB, IMMED16\\
  movui & movui rB, IMMED16\\
  srl & srl rC, rA, rB\\
  srli & srli rC, rA, IMMED5\\
  sra & sra rC, rA, rB\\
  srai & srai rC, rA, IMMED5\\
  sll & sll rC, rA, rB\\
  slli & slli rC, rA, IMMED5\\
  jmp & jmp rA\\
  br & br LABEL\\
  beq & beq rA, rB, LABEL\\
  bne & bne rA, rB, LABEL\\
  blt & blt rA, rB, LABEL\\
  ble & ble rA, rB, LABEL\\
  bgt & bgt rA, rB, LABEL\\
  bge & bge rA, rB, LABEL\\
  bltu & bltu rA, rB, LABEL\\
  bleu & bleu rA, rB, LABEL\\
  bgtu & bgtu rA, rB, LABEL\\
  bgeu & bgeu rA, rB, LABEL\\
  call & call LABEL\\
  ret & ret\\
\end{tabular}
}

\end{document}


\documentclass[10pt]{article}
\usepackage[top=1in,bottom=1.1in,left=.8in,right=.8in]{geometry}
\usepackage[T1]{fontenc}
\usepackage[ansinew]{inputenc}
\usepackage{graphicx}
\usepackage{multirow}


\renewcommand{\familydefault}{\sfdefault}

\begin{document}
%\maketitle

\hspace{-5mm}
\begin{minipage}{0.65\linewidth}
  \textbf{
      \hspace{-3mm}
      {\Large COMP 231-01}\\
      {\Large Introduction to Computer Organization}\\
      {\Large Homework 4}}
\end{minipage}
\begin{minipage}{0.35\linewidth}
  \includegraphics[scale=.3]{../../logos/rhodes-logo.jpg}
\end{minipage}

\noindent{\bf Due Date: Thursday, October 31}\\
%\noindent{\bf Due Date: Wednesday, November 6th.}\\
%\noindent{\bf Due Date: Monday, April 1st.}\\
%\noindent{\bf Due Date: Tuesday, November 8th.}\\
%\noindent{\bf Due Date: Wednesday, March 22nd.}\\
%\noindent{\bf Due Date: Wednesday, April 5th.}\\
%\noindent{\bf Due Date: Friday, November 10th.}\\


\begin{enumerate}

\setlength\itemsep{10mm}

\item Convert the following C program fragment to NIOS assembly
  language:

\begin{verbatim}
   if (x < y) {
       y = x+1;
   else if (x == 3)
       y = x+2;
   else
       y = 0;
\end{verbatim}

Assume that {\tt x} is located in register {\tt r10} and {\tt y} is in
{\tt r11}.


\item Convert the following C program fragment to NIOS assembly
  language:

\begin{verbatim}

  int c, n, fact, sumoffact;
  n = 10;
  fact = 1;
  sumoffact = 0;

  for (n=1; n<= 10; n++) {
    for (c=1; c<=n; c++) {
      fact = fact * c;
    }
    sumoffact += fact;
  }

\end{verbatim}

Assume that {\tt c} is located in register {\tt r10},  {\tt n} is in
{\tt r11}, and that {\tt fact} and {\tt sumoffact} are in memory
locations 1000 and 2000, respectively.

\end{enumerate}


\end{document}


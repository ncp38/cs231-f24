\documentclass[10pt]{article}
\usepackage[top=1in,bottom=1.1in,left=.8in,right=.8in]{geometry}
\usepackage[T1]{fontenc}
\usepackage[ansinew]{inputenc}
\usepackage{graphicx}
\usepackage{multirow}
\usepackage{url}

\renewcommand{\familydefault}{\sfdefault}

\begin{document}
%\maketitle

\hspace{-5mm}
\begin{minipage}{0.65\linewidth}
  \textbf{{\Large COMP 231\\
      Introduction to Computer Organization\\Lab 2}}
\end{minipage}
\begin{minipage}{0.35\linewidth}
  \includegraphics[scale=.3]{../../logos/rhodes-logo.jpg}
\end{minipage}

\vspace{.25in}

This lab will consist of a Quartus project folder, submitted as a
single ZIP file, via Canvas (\url{http://canvas.rhodes.edu/}).
{\bf You must submit a .zip file in the following format, otherwise
  you will lose points.} When you submit your project, rename the
project folder to <{\em name\_lab1}>, where the name is your Rhodes
email ID. Then, ZIP the folder up and upload it to Canvas. Do not
use any other compression/archiving format other than ZIP.

\section*{Hardware Adder}

For this lab, you will be creating a circuit which takes a pair of two-bit
values from the switches on the DE2-115, adds them together, creating a 3-bit
value which will be displayed on the LED lights. Create a new Quartus project,
{\tt adder}, in your {\em name}{\tt\_lab2} folder.

The inputs are signified as $a_1 a_0$ and $b_1 b_0$. The output is
signified as $c_2 c_1 c_0$. You are computing the following expression
assuming unsigned binary integer representation:

\begin{center}
\begin{tabular}[h]{cccc}
    &       & $a_1$ & $a_0$ \\
  + &       & $b_1$ & $b_0$ \\\hline
    & $c_2$ & $c_1$ & $c_0$ \\
\end{tabular}
\end{center}

\section*{Analysis}

You will need to compute a truth table that will help you determine
the values for $c_2$, $c_1$, and $c_0$. From this truth table, derive
minimal sum of products equations for each of the output functions.
You will need to turn in this analysis on the due date in class on
paper. {\bf You should not use a design based on the ripple-carry
  adders discussed in class.} Your solution should perform the
addition directly in terms of all of the input signals, without
introducing the propagation delay inherent in a ripple-carry system.

\section*{Circuit Design}

Using Quartus II, implement the combinational circuits that you
derived. For the addition circuit, you may only use {\bf and gates}, {\bf or
gates}, or {\bf inverters/not gates}. You may not use any built-in addition circuits
from the Quartus library. Compile and deploy your design to the DE2
board.  Verify that the design and implementation is correct and
works. It will be easier to use the DE2-115 pin assignments file that
can be found on Canvas, than the physical pin names as used in the
tutorial. You can use the logical names {\tt LEDR[0], LEDR[1]}, and
{\tt SW[0]}, etc. as the pin names associated with the switches and
LED lights. {\bf Your submission should use multiple blocks (one for
  each output).} There is a tutorial document on Canvas about using
blocks within the Quartus schematic editor.



\section*{Submission}

When you have completed this entire exercise and have a functioning
program, submit your lab 2 project folder as specified above to Canvas.

\end{document}


\documentclass[10pt]{article}
\usepackage{amssymb}
\usepackage{listings}
\usepackage[top=.7in,bottom=1.1in,left=.8in,right=.8in]{geometry}
\usepackage{graphicx}
\usepackage[T1]{fontenc}
\usepackage[ansinew]{inputenc}
\usepackage{url}
\usepackage{amsmath}

\renewcommand{\familydefault}{\sfdefault}

\begin{document}
\hspace{-5mm}
\begin{minipage}{0.65\linewidth}
  \textbf{{\Large COMP 231\\Lab 5}}
\end{minipage}
\begin{minipage}{0.35\linewidth}
  \includegraphics[scale=.3]{../../logos/rhodes-logo.jpg}
\end{minipage}

\vspace{.4in}

\noindent For this lab exercise, you will create a NIOS II CPU-based
system as described below. Submit your entire Quartus project folder
as a ZIP file. Do not use any other archive/compression
formats other than ZIP. \\

%\noindent{\bf Due Date: 11.59.59pm, Friday. Nov. 11th 2016}. No late
%\noindent{\bf Due Date: 11.59.59pm, Friday. Apr. 7th 2017}. No late
%\noindent{\bf Due Date: 11.59.59pm, Friday. Nov. 17th 2017}. No late
%\noindent{\bf Due Date: 11.59.59pm, Tuesday. Nov. 27th 2018}. No late
%\noindent{\bf Due Date: 11.59.59pm, Friday. Apr. 5th 2019}. No late
%\noindent{\bf Due Date: 11.59.59pm, Wednesday. Nov. 13th 2019}. No late
%work will be accepted.

\section{Seven-Segment Display support}

For the first part of the project, start by copying your {\tt lights} Nios
project from the Lab 4 tutorial to a new folder. We will be adding support for
the seven-segment display devices and driving them with a software-based
lookup table. First, we will need to add the hardware support and connections
for the SSDs.

Return to the Qsys tool (see page 11) and add two new PIO (Parallel I/O) ports with
a data path width of 8-bits and as an output port (Just like you did
for green LEDs). Rename the first HEXA and the second HEXB. Make
connections in Qsys to the Avalon bus in the same way as you did for
the LED PIO port (clock, reset, etc.).

Next, you will need to re-assign base addresses for the new
ports. This may change the earlier set values for the switches and LED
ports. Take note of the memory-mapped addresses for the switches,
LEDS, HEXA, and HEXB.  You will need this information for the next
section.

Lastly, ensure that you export the output for the HEXA and HEXB ports,
the same way you configured the switches and LEDs. Generate the HDL
from Qsys and head back to Quartus.

\section{VHDL Modification}

Now we need to connect the output wires from HEXA and HEXB on the NIOS
processor to the input wires for HEX0 and HEX1 SSD hardware on the
DE2. To do this, we will need to modify the VHDL file, {\tt
  lights.vhd}, that we had created during the tutorial.

\noindent Qsys outputs a sample VHDL file to instantiate the Nios processor,
given the names for the devices that you configured. You should verify
the signal names in the sample file, which should be {\tt
  nios\_system/nios\_system\_inst.vhd}.\\

\noindent {\bf In the {\tt lights.vhd} file, you need to make minor changes to several parts:}\\

\noindent {\bf PORT Section:}  In this portion, we need to declare the interface
to the DE2 components. Add lines to the VHDL file like so:

\begin{verbatim}
    HEX0 : OUT STD_LOGIC_VECTOR (6 DOWNTO 0);
    HEX1 : OUT STD_LOGIC_VECTOR (6 DOWNTO 0)
\end{verbatim}

Ensure that all lines end in a semicolon, except for the last. \\

\noindent {\bf ARCHITECTURE Section:}  This section should match the information
from {\tt nios\_system\_inst.vhd}. Add new {\tt SIGNAL} lines that
correspond to the 8-bit output ports. They should have names like {\tt
  hexa\_export} and {\tt hexb\_export} and be placed in the {\tt PORT}
portion of the {\tt ARCHITECTURE} block.

Next, we need to declare two 8-bit buses to connect the HEXA port to
HEX0 and HEXB to HEX1. Add the following lines to the {\tt
  ARCHITECTURE} block after the {\tt END COMPONENT;} line:

\begin{verbatim}
   SIGNAL HEXA : std_logic_vector(7 downto 0);
   SIGNAL HEXB : std_logic_vector(7 downto 0);
\end{verbatim}

\noindent {\bf BEGIN Section:} Lastly we need to bring everything together and
make all of the connections. Right after the {\tt BEGIN} statement,
add a sequence of mappings to connect the HEXA and HEXB wires to the
HEX0 and HEX1 SSDs. For example:

\begin{verbatim}
  HEX0(0) <= HEXA(0);
\end{verbatim}

This statement attaches the low-order bit of the 8-bit HEXA wire to
the low-order bit of the HEX0 SSD unit. Connect wires from HEXA to
HEX0 and HEXB to HEX1. Both HEX0 and HEX1 only have seven input wires,
so you only need to worry about wires $0\ldots6$.

Lastly, in the {\tt PORT MAP} section, add mappings for {\tt
  hexa\_export} to map to {\tt HEXA} and {\tt hexb\_export} to {\tt
  HEXB}. {\bf Ensure that you have been careful with your commas and
semicolons} and save your file. Back in Quartus, re-compile the entire
project. If you are successful, you are done with the hardware setup
and are ready to work on the software side of the project.

\section{Assembly Language Programming}

The final part of this assignment involves writing an assembly
language program that reads an 8-bit value from the switch input and
displays the value in hexadecimal format on the SSD units (and on the
green LEDs). Start by working from the {\tt lights.s} file. As in the
tutorial assembly code, create {\tt .equ} macros for the addresses of the PIO
ports. Create one each for {\tt leds, switches, hexa,} and {\tt
  hexb}. These may have changed with the addtion of the new PIO ports.

To display a value on the seven-segment display, you need to issue a
{\tt stbio} instruction to the address from the Qsys layout. The value
you write to the address is a 7-bit value which corresponds to each
segment of the display. Recall that the SSD device uses
negative-asserted true logic, so a 0 bit value activates an LED
segment and a 1 bit value deactivates a segment. For example, if we
wanted to make an 8 display, we would write 7 zeroes to the
display: {\tt 0x00}. To make only the top light segment display
(segment 0), you would write all ones, except for the lowest bit (bit
0): {\tt 0xfe} or {\tt 0x7e} (the highest-order bit is ignored, since
the SSD inputs are only seven-bit values).\\

\noindent {\bf Lookup Table:}  Rather than using the hardware
circuitry that we created in Lab 2, we want to use a software lookup
table to display the correct value on the SSD. To implement this, we
will use a label, {\tt lut:}, to mark the beginning of some constant
data in our assembly language program.


To understand how a lookup table is implemented in assembly language,
assume that we wanted to have a three-element array constant, {\tt a},
with {\tt a[0] = 5, a[1] = 6,} and {\tt a[2] = 7}. We would arrange
this in our assembly language program like this:

\begin{verbatim}
.data
a:
.word 5
.word 6
.word 7
.end
\end{verbatim}

This code is placed after the end of our assembly program, before the
{\tt .end} marker. The label {\tt a:} marks the address of the first
element of the array (index 0). Each element is stored in a four-byte
word, so to compute {\tt a[2]}, we would use the base address stored
in {\tt a}, and add the size of each element (4 bytes) times the
number of elements to skip over (2): $ a_2 = $ {\tt Mem[} $a +
(4\times 2)${\tt ]}

Construct a 16-element array, one for each possible 4-bit combination
of values to display on an SSD. The value of each array element should
be a 32-bit word, with the lowest 7-bits containing the bit pattern to
illuminate the correct segments for the corresponding 4-bit
hexadecimal value. For example, the value of {\tt lut[8]} should
contain the 7-bit pattern ({\tt 0x00}) to illuminate all segments on
the display.

Once you have completed your array, modify the main loop driver to write the
8-bit value from the switches to both HEXA and HEXB. You will have to read the
8-bit value into two registers, use these values to determine the offsets into
the lookup table, load the pattern values from the LUT and write them each to
the correct SSD, as well as the LEDs.

As an example, let's say that the value read from the switches is {\tt
  1111 0001} (i.e. {\tt 0xf1}). You need to move {\tt 0001} into one register and {\tt
  1111} into another. Next, load the value at ${\tt lut} + 4 \times
1$ into a new register and ${\tt lut} + 4 \times 15$ into
another. These registers should now contain the 7-bit patterns that
can be written to the SSD using the {\tt stbio} instruction.

\section*{Submission}

Make sure you submit an entire project directory to Canvas. This
should contain all of the Quartus II files. If you do not submit an
fully functioning project, you will lose points. {\bf Double check
  your work if you make changes to files and/or file locations.}

\end{document}

\documentclass[10pt]{article}
\usepackage{amssymb}
\usepackage{listings}
\usepackage[top=.7in,bottom=1.1in,left=.8in,right=.8in]{geometry}
\usepackage{graphicx}
\usepackage[T1]{fontenc}
\usepackage[ansinew]{inputenc}
\usepackage{url}
\usepackage{amsmath}

\renewcommand{\familydefault}{\sfdefault}

\begin{document}
\hspace{-5mm}
\begin{minipage}{0.65\linewidth}
  \textbf{{\Large COMP 231\\Lab 4}}
\end{minipage}
\begin{minipage}{0.35\linewidth}
  \includegraphics[scale=.3]{../../logos/rhodes-logo.jpg}
\end{minipage}

\vspace{.5in}

\noindent For this lab exercise, you will create a NIOS II CPU-based
system as described below. Submit your entire Quartus project folder
as a ZIP file. Do not use any other archive/compression
formats other than ZIP. \\

%\noindent{\bf Due Date: 11.59.59pm, Monday, Oct. 28, 2019}. No late
%work will be accepted.
%\noindent{\bf Due Date: 11.59.59pm, Friday, Mar. 22, 2019}. No late
%work will be accepted.
%\noindent{\bf Due Date: 11.59.59pm, Wednesday, Oct. 31, 2018}. No late
%work will be accepted.
%\noindent{\bf Due Date: 11.59.59pm, Tuesday, Oct. 31, 2017}. No late
%work will be accepted.
%\noindent{\bf Due Date: 11.59.59pm, Monday, Mar. 20th, 2017}. No late
%work will be accepted.

\section{Nios II Tutorial}

Read through and follow the exercise specified in the document, {\em
  Introduction to the Altera Qsys System Integration Tool}, available on
the course website and Canvas. Create the {\tt lights} Quartus project in your {\tt lab4} folder. {\bf Do
not use other filenames or you will be in trouble}.

In this exercise you will use the Qsys tool to construct a Nios CPU with a
clock, some on-chip memory, a parallel input port, and a parallel output port.
Once you have the system configured in Qsys, you should generate the hardware
description language (HDL) that Quartus will use to compile the system.  Next,
you will need to write a small program in VHDL to map the Nios system to the
actual devices on the DE2-115.

The tutorial contains some sections that we are not using for our assignment.
We only need one HDL component -- we will be using VHDL, so the Verilog
section (5.1.1) can be skipped. Secondly, we are only interested in programming
the Nios CPU with assembly language. You can also safely skip section 7.2,
which is a C program and not assembly.

You should ensure that you have added both the Qsys generated HDL file as well
as your VHDL code into the main Quartus program. Next, compile the project in
Quartus. Remember to add your pin assignements from the {\tt DE2\_115.qsf}
file. If you get an error like: ``{\tt Error: Can't generate netlist output
  files$\ldots$}'', you will need to go to the {\em Settings} dialog box from
the Assignments menu. Select the category for {\em EDA Tool Settings},
subcategory {\em Simulation}. On this page, change the {\em Tool Name} option
to {\tt <None>}.

Once you have compiled your project, create the simple Nios assembly language
program from the tutorial and put it into your {\em name}{\tt \_lab4}
directory. Create a new project using the Altera Monitor Program that uses your
Nios system and assembly language program.

Next, download the Nios system to the DE2 using either the Quartus Programmer
tool or the Altera Monitor Program. Verify that everything works correctly by
stepping through your program in the monitor.

\section*{Submission}

Make sure you submit an entire project directory to Canvas. This should contain
all of the Quartus II files. If you do not submit an fully functioning project,
you will lose (many) points.

\end{document}

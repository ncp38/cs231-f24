\documentclass[10pt]{article}
\usepackage[top=1in,bottom=1.1in,left=.8in,right=.8in]{geometry}
\usepackage[T1]{fontenc}
\usepackage[ansinew]{inputenc}
\usepackage{graphicx}
\usepackage{multirow}
\usepackage{url}

\renewcommand{\familydefault}{\sfdefault}

\begin{document}
%\maketitle

\hspace{-5mm}
\begin{minipage}{0.65\linewidth}
  \textbf{{\Large COMP 231\\
      Introduction to Computer Organization\\Lab 1}}
\end{minipage}
\begin{minipage}{0.35\linewidth}
  \includegraphics[scale=.3]{../../logos/rhodes-logo.jpg}
\end{minipage}

\vspace{.25in}

This lab will consist of a Quartus project folder, submitted as a single ZIP
file, via Canvas (\url{http://canvas.rhodes.edu/}).  {\bf You must submit a
.zip file.} \\

\noindent{\bf Important information for using the lab workstations:}
\begin{enumerate}
  \item Start by creating a new folder on your desktop named {\em 231}.

  \item Create a new folder inside of the 231 folder called {\em lab1}. This is where you will save your tutorial.

  \item Follow the tutorial. When it prompts you where you want to save the
    folder -- {\bf You must change it to a different directory.} Instead of the
    {\tt C:\textbackslash intelFPGA\_lite\textbackslash18.1}$\ldots$ folder,
    switch it to the {\em lab1} folder you created under the Desktop.

  \item Please do not save any work in the {\tt intelFPGA\_lite} folder or
    anywhere else on the {\tt C:\textbackslash} drive.

  \item When you have completed the project, right click on your {\em lab1} folder and use the {\tt 7-Zip} menu
    to save as {\tt lab1.zip}.

  \item Save your entire Desktop/231 folder to a flash drive (in the file explorer)

  \item Upload your .ZIP file to Canvas.
\end{enumerate}

You are encouraged to use one of the six workstations in the hardware
lab to complete this assignment. On your personal computer, you may also install the Digital software package to simulate your designs (\url{https://github.com/hneemann/Digital}) or the Altera tools (ask me for more details).

\section{Quartus Tutorial}

Once you have successfully logged into the hardware lab computer,
read through and follow the exercise specified in the {\em
  Quartus II Introduction Using Schematic Designs} PDF (on
the course website, listed under Lab 1). Create the {\tt introtutorial} Quartus project in your {\em
  name}{\tt\_lab1} folder.


In this tutorial, you will implement a simple exclusive-OR (XOR)
circuit and load it on the FPGA board. This circuit will use the
switches (SW[0] and SW[1]) on the DE2-115 board to turn on and off the LED
light on the board according to the XOR truth table.

Some notes:

\begin{itemize}
  \item The pin assignment procedure in section 7 is suboptimal. The pins
    for the DE2-115 are incorrect and the entire process is overly complicated for
    the other FPGA assignments in the class. To remedy this, you should label your
    inputs in the XOR circuit with the names {\tt SW[0]} and {\tt SW[1]}. Also label your
    output {\tt LEDG[0]} or {\tt LEDR[0]} depending on whether you like red or green LEDs.
    Next, from the {\it Assignments} menu, select {\it Import Assignments...} and
    browse to the {\tt DE2\_115.qsf} file (on the course website, listed under Lab 1). This should leave you
    with valid pin assignments.

  \item You do not need to perform the simulation step in section 8 of the PDF.

  \item When starting the programmer, you may have to add the compiled Quartus
    (.sof) file, which can be found in the {\tt output\_files} folder in your
    {\tt introtutorial} project.
\end{itemize}


\section*{Submission}

When you have completed this entire exercise and have a functioning
program, submit your lab 1 project folder as specified above to Canvas.

\end{document}

